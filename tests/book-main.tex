\documentclass[output=book
  ,colorlinks
%   ,collection
  ,showindex
%   ,draftmode
%   ,openreview
%   ,nobabel
%   ,booklanguage=italian
%   ,oldstylenumbers
%   ,multiauthors
    ,tblseight
%   ,booklanguage=german
  ]{langscibook}

\usepackage{langsci-tbls}
\usepackage{langsci-linguex}
\usepackage{langsci-gb4e}
\usepackage{langsci-optional}
% \usepackage{langsci-tobi}
\usepackage[danger]{langsci-lgr}
\usepackage{langsci-conversationanalysis}
\usepackage{langsci-plots}
\usepackage{langsci-avm}

\usepackage{lipsum}
\usepackage{multicol}
\usepackage{minibox}
\usepackage[linguistics,edges]{forest}
\usepackage{tikz}
\usepackage{pgfplots}

\renewcommand{\lsISBNhardcover}{999-3-123456-99-9}
\renewcommand{\lsISBNhardcover}{999-3-123456-99-9}
\renewcommand{\lsISBNsoftcover}{999-3-123456-99-9}
\renewcommand{\lsISBNdigital}{978-3-946234-65-4}
\renewcommand{\lsID}{999}
\renewcommand{\publisherstreetaddress}{LangSci\\Galactic Highway 42\\Olympus Mons}


\addbibresource{localbibliography.bib}


\BackBody{
What causes a language to be the way it is? Some features are universal, some are inherited, others are borrowed, and yet others are internally innovated. But no matter where a bit of language is from, it will only exist if it has been diffused and kept in circulation through social interaction in the history of a community. This book makes the case that a proper understanding of the ontology of language systems has to be grounded in the causal mechanisms by which linguistic items are socially transmitted, in communicative contexts. A biased transmission model provides a basis for understanding why certain things and not others are likely to develop, spread, and stick in languages. Because bits of language are always parts of systems, we also need to show how it is that items of knowledge and behavior become structured wholes. The book argues that to achieve this, we need to see how causal processes apply in multiple frames or 'time scales' simultaneously, and we need to understand and address each and all of these frames in our work on language. This forces us to confront implications that are not always comfortable: for example, that "a language" is not a real thing but a convenient fiction, that language-internal and language-external processes have a lot in common, and that tree diagrams are poor conceptual tools for understanding the history of languages. By exploring avenues for clear solutions to these problems, this book suggests a conceptual framework for ultimately explaining, in causal terms, what languages are like and why they are like that.
}

\dedication{Für Alma, Ariel, Block, Frau Brüggenolte, Chopin, Christina, Doro, Edgar, Elena, Elin, Emma, den ehemaligen FCR Duisburg, Frida, Gabriele, Hamlet, Helmut Schmidt, Henry, Ian Kilmister, Ingeborg, Ischariot, Jean-Pierre, Johan, Kurt, Lemmy, Liv, Marina, Martin, Mats, Mausi, Michelle, Nadezhda, Herrn Oelschlägel, Oma, Opa, Pavel, Philly, Sarah, Scully, Stig, Tania, Tante Klärchen, Tarek, Tatjana, Herrn Uhl, Ullis schreckhaften Hund, Vanessa und so. Wenn das schonmal klar sein würde.}


\title{Test for the langsci-* packages}
\author{Lang Uage and Science and Press}
\renewcommand{\lsYear}{1999}


\renewcommand{\lsSeries}{tbls}

\lsCoverTitleSizes{40pt}{15mm}
% % \usepackage{../langsci-branding}
\begin{document}
\renewcommand{\lsImpressumExtra}{Manuscript submitted in fulfillment of entering the Galactical Hall of Fame.}


\maketitle
\tableofcontents
\mainmatter

%
\part{Part test}
\chapter{Tests}
\section{langsci-gb4e}
\subsection{Environments and syntactic sugar}
\input{gb4e-tests/beginexe}
% % % \input{gb4e-tests/exbegin}
\input{gb4e-tests/subex}
\input{gb4e-tests/xlist}
\input{gb4e-tests/subsubex}
\input{gb4e-tests/exi}
\input{gb4e-tests/ea}
\input{gb4e-tests/judgment}
\input{gb4e-tests/parbox}
\input{gb4e-tests/glttests}
\input{linguex-tests/linguex}
% 
% 
\subsection{Glosses}
\input{gb4e-tests/gll}
\input{gb4e-tests/glll}
\input{gb4e-tests/gllll}
\input{gb4e-tests/linebreak}
\input{gb4e-tests/exewidth}

\subsection{Footnotes}
\input{gb4e-tests/footnote}
\input{gb4e-tests/footnotemark}
\input{gb4e-tests/exinfn}
\subsection{Metadata}
\input{gb4e-tests/languageinfo}
\subsection{Boxing}
\input{gb4e-tests/xbox}
\subsection{Jambox}
\input{gb4e-tests/jambox}
\subsection{XPs}
\input{gb4e-tests/bars}
\input{gb4e-tests/crossrefref}
\input{gb4e-tests/eal}

\subsection{Styles for source line}
\input{gb4e-tests/nontypo-ex}
\input{gb4e-tests/typo-ex}
\input{gb4e-tests/underlinedexample}

\section{AVM}
\input{avm-tests/langsci-avm.tex}

\section{Maths}
\input{math-tests/formulae}

\section{Trees}
\input{forest-tests/simpletrees}

\section{Fonts}
\input{font-tests/fonts.tex}
\input{font-tests/tiebars.tex}

\section{Tables}\label{sec:tables}
\input{table-tests/general}
\input{table-tests/footnotes}

\section{Crossrefs}
\input{crossref-tests/general}

\section{Bibliography}
\input{bib-tests/cite}

\printbibliography[notkeyword={techreport},notkeyword={website},title={References}]

next thing up is list of standards, included with \verb+keywords={standard},+ in the bib file.

\printbibliography[keyword={standard},title={Standards}]



next thing up is list of websites, included with \verb+keywords={website},+ in the bib file.

\printbibliography[keyword={website},title={Websites}]



\section{Index}
\input{index-tests/indexes}

\section{Diagrams}
\input{diagram-tests/plots}

\section{Floats}
\input{float-tests/floats.tex}
\section{Conversation analysis}
\input{conversationanalysis-tests/ca.tex}


% % \input{intonation-tests/tobi} % this package is no longer being developed
%
\section{Langsci-lgr tests}
\input{lgr-tests/lgr}

\section{Textbooks tests}
% \input{tbls-tests/mdframed} %% Tests for \usepackage[mdframed]{langsci-tbls}
\input{tbls-tests/tcolorbox}

\section{Syntactic sugar}
%%%%%%%%%%%%%%%%%%%%%%%%%%%%%%%%%%%%%%%%%%%%%%%%%%%%%%%%%%%%%%%%%%%%%
%%      File: langsci-optional.sty
%%    Author: Language Science Press (http://langsci-press.org)
%%      Date: 2016-01-16 16:47:43 UTC
%%   Purpose: This file contains useful, but not essential,
%%            macros for books using langscibook.cls
%%  Language: LaTeX
%%   Licence:
%%%%%%%%%%%%%%%%%%%%%%%%%%%%%%%%%%%%%%%%%%%%%%%%%%%%%%%%%%%%%%%%%%%%%

\usepackage{pbox}   % boxes with maximum width
% Heiko Oberdiek
% http://tex.stackexchange.com/questions/136644/vertical-space-in-interaction-with-figure-center-environment
\newcommand{\oneline}[1]{%
  \begingroup
    \sbox0{\ignorespaces#1\unskip}%
    \leavevmode
    \ifdim\wd0>\linewidth
      \hbox to\linewidth{%
        \hss\resizebox{\linewidth}{!}{\copy0 }\hss
      }%
    \else
      \copy0 %
    \fi
  \endgroup
}

\newcommand{\centerfit}[1]{%
  \begingroup
    \sbox0{\ignorespaces#1\unskip}%
    \leavevmode
    \ifdim\wd0>\linewidth
      \hbox to\linewidth{%
        \hss\resizebox{\linewidth}{!}{\copy0 }\hss
      }%
    \else
      \centerline{\copy0 }%
    \fi
  \endgroup
}

% Helps to fit verbatim onto one line:
% http://tex.stackexchange.com/questions/140593/shrinking-verbatim-text/
\usepackage{fancyvrb}
\newenvironment{fitverb}
 {\SaveVerbatim{rlwv}}
 {\endSaveVerbatim
  \sbox0{\BUseVerbatim{rlwv}}
  \begingroup\center % don't add indentation
  \ifdim\wd0>\linewidth
    \resizebox{\linewidth}{!}{\copy0}%
  \else
    \copy0
  \fi
  \endcenter\endgroup}

% \VerbatimFootnotes %breaks roman numbering for examples in footnotes


% http://tex.stackexchange.com/questions/73464/inserting-rtl-text-in-verbatim-environment?rq=1
% verbatim with RTL text

%\DefineVerbatimEnvironment{rtlverbatim}{Verbatim}{commandchars=+\[\]}

%add intonation bars over morphemes or words
\newcommand{\intline}[2]{\settowidth{\LSPTmp}{#2}\raisebox{#1pt}{\parbox{.1mm}{\rule{\LSPTmp}{.5pt}}}#2}

%add rising or falling intonation
\newcommand{\dline}[3]{%
  \parbox{.1mm}{\begin{picture}(0,0)%
        \put(0,#1){\line(#2,-1){#3}}%
        \end{picture}%
  }%
}

%% rotated table headers
% create lengths
\newlength{\rotheight}
\newlength{\rotwidth}

\newcommand{\rotatehead}[2][1cm]{
%width is the width of the parbox
%height is the buffer space used to vertically stretch the headere
\setlength{\rotwidth}{#1}
\setlength{\rotheight}{.85\rotwidth}
  \begin{rotate}{33}~ %nbsp shifts the content away from the line underneath
   \parbox{\rotwidth}{\raggedright #2}
  \end{rotate}%
  \rule{0pt}{\rotheight} %add zero width rule to get the right table height
}

% example metadata

\newcommand{\langinfo}[3]{{\upshape #1\il{#1}~(%
\ifx\\#2\\%
\else%
#2;
\fi%
#3)}\nopagebreak[4]\ignorespaces}

\newcommand{\langinfoverb}[3]{{\upshape #1~(% langinfo without index marker
\ifx\\#2\\%
\else%
#2;
\fi%
#3)}\nopagebreak[4]\ignorespaces}

\newcommand{\fitpagewidth}[1]{
  \resizebox{\textwidth}{!}{#1}
}
\newcommand{\fittable}[1]{\resizebox{\textwidth}{!}{#1}}

\usepackage{array}
\newenvironment{widetabular}[1][1]
  {\tabularx{#1\textwidth}}
  {\endtabularx}


% Vowel chart tikz commands
\newcommand{\aeiou}{%
	\node at (1.5,0) (a) {a};
	\node at (0,3) (i) {i};
	\node at (3,3) (u) {u};
	\node at (0.5,1.5) (e) {e};
	\node at (2.5,1.5) (o) {o};
}
\newcommand{\aeiouEO}{%
	\node at (1.5,0) (a) {a};
	\node at (0,3) (i) {i};
	\node at (3,3) (u) {u};
	\node at (0.25,2) (e) {e};
	\node at (2.75,2) (o) {o};
	\node at (0.75,1) (E) {ε};
	\node at (2.25,1) (O) {ɔ};
}

\usepackage{tabularx} 
%no hyphenation left alingned
\newcolumntype{Q}{>{\raggedright\arraybackslash}X}
%no hyphenation right aligned
\newcolumntype{Y}{>{\raggedleft\arraybackslash}X}
%no hyphenation centered
\newcolumntype{C}{>{\centering\arraybackslash}X}
%no hyphenation fixed width
\newcolumntype{L}[1]{>{\raggedright\let\newline\\\arraybackslash\hspace{0pt}}m{#1}}
%no hyphenation centered fixed width
\newcolumntype{Z}[1]{>{\centering\let\newline\\\arraybackslash\hspace{0pt}}m{#1}}
%no hyphenation right aligned fixed width
\newcolumntype{R}[1]{>{\raggedleft\let\newline\\\arraybackslash\hspace{0pt}}m{#1}}

% Underlining in gb4e-example Environments. Usual underlining commands that span multiple words do not work, because gb4e would parse it as one word.
% Example: \underline{My example phrase} should become \ulp{My}{~~~~~} \ulp{example}{~~~~~} \ule{phrase}
% Note: The 2nd Argument of the \ulp command is filled in by experience - if you are not familiar with the command, you should experiment a bit. Usually, five tildes are enough, but be sure to check the outcome.
% \ule is meant to be the last word in a phrase that is underlined. Therefore, \ule does not have an extra length.
\usepackage[normalem]{ulem}
\usepackage{calc}
\newlength{\fulllength}
\newcommand{\ulp}[2]{%#1: stuff to underline, #2: extra length to skip the whitespace between to components
  \settowidth{\LSPTmp}{#1}%
  % several boxes are need to assure that words with ascending and descending letters are underlined at the same
  % level, leading to the impression of a continuous stroke
  \parbox[t]{\LSPTmp}{%restrict first box to the length of first argument
      \settowidth{\fulllength}{\parbox{\LSPTmp}{~}\parbox{#2mm}{~}} %inner box is larger than outerbox, so underlining will extend beyond length of outer box
      %             align parbox to bottom
      %              |           mbox to prevent hyphenation
      \uline{\parbox[b]{\fulllength}{\mbox{#1}}}%
    }%
}

\newcommand{\ule}[1]{%#1: stuff to underline, no extra length
  \ulp{#1}{0}%
  }

\newlength{\fullllength}
\newcommand{\soutp}[2]{%#1: stuff to underline, #2: extra length to skip the whitespace between to components
  \settowidth{\LSPTmp}{#1}%
  % several boxes are need to assure that words with ascending and descending letters are underlined at the same
  % level, leading to the impression of a continuous stroke
  \parbox[t]{\LSPTmp}{%restrict first box to the length of first argument
      \settowidth{\fullllength}{\parbox{\LSPTmp}{~}\parbox{#2mm}{~}}%inner box is larger than outerbox, so underlining will extend beyond length of outer box
      %             align parbox to bottom
      %              |           mbox to prevent hyphenation
      \sout{\parbox[b]{\fullllength}{\mbox{#1}}}%
    }%
}

\newcommand{\soute}[1]{%#1: stuff to underline, no extra length
  \soutp{#1}{0}%
  }

\newcommand{\longrule}{\rule{1em}{.3pt}}
\usepackage{colortbl}
\newcommand{\shadecell}{\cellcolor{black!20!white}}

% vertical alignment of  numbered  example
\newcommand{\eabox}[2][-.7\baselineskip]{
  \ea
    \parbox[t]{.8\textwidth}{
      \vspace{#1}
      #2
     }
  \z
}
\newcommand{\exbox}[2][-.7\baselineskip]{
  \ex
    \parbox[t]{.8\textwidth}{
      \vspace{#1}
      #2
     }
}

%fix \verb error in biblatex
% \makeatletter
% \def\blx@maxline{77}
% \makeatother

\usepackage{todonotes}
\newcommand{\rephrase}[2]{{\color{yellow!30!black}#2}\todo{replaced `#1'}}
\newcommand{\missref}[2][]{\todo[#1]{missing reference #2}}

\newenvironment{indentquote}[1]%
  {\list{}{\leftmargin=#1\rightmargin=0pt}\item[]}%
  {\endlist}


\newcommand{\phonrule}[3]{#1 $\to$ #2 / #3}
\newcommand{\featurebox}[1]{$\left[\begin{tabular}{>{\scshape}c}#1\end{tabular}\right]$}



%connect two elements with lines
\newcommand{\connect}[2]{%
  \tikz[overlay,remember picture]{%
    \draw[-,thick] (#1) -- (#2) node   {};  %
  }
}

%%%%%%%%%%%%%%%%%%%%%%%%%%%%%%%%%%%%%%%%%%%%%%%%%%%%%%%%%%%%%%
%%%% Experimental feature for pointing out moving things. %%%%
%%%% Buggy? Write to: kopeckyf@hu-berlin.de               %%%%
%%%%%%%%%%%%%%%%%%%%%%%%%%%%%%%%%%%%%%%%%%%%%%%%%%%%%%%%%%%%%%
\usetikzlibrary{arrows,arrows.meta}
\newcounter{lsConnectTempGroup}
\NewDocumentCommand\ConnectTail{m O{\thelsConnectTempGroup}}{%read: mandatory arg #1, optional argument #2 with the current group counter as its default value.
    \edef\lsConnectTempPosition{#2}%\edef expands the argument, which means reading the current value of the counter.
    {\tikz[remember picture,
           anchor=base, baseline,
           inner xsep=0pt,
           inner ysep=-.5ex]\node (ConnectTempTail\lsConnectTempPosition) {\strut{}#1};}%\strut for baseline
}
\NewDocumentCommand\ConnectHead{s O{1ex} m O{\thelsConnectTempGroup}}{%read: star #1, optional argument (distance of arrow from text= std. one x-height), mand. arg. (node text), optional argument #2, the group specifier
    \edef\lsConnectTempPosition{#4}%
    \stepcounter{lsConnectTempGroup}%We have a match, let's update the group counter
    {\tikz[remember picture,
           anchor=base, baseline,
           inner xsep=0pt,
           inner ysep=-.5ex] \node (ConnectTempHead\lsConnectTempPosition) {\strut{}#3};%
     \tikz[remember picture] \draw[% we have a tail and a head, let's bring them together
                                \IfBooleanTF#1{{Triangle[]}-}{-{Triangle[]}},% Check if the starred version is used. The starred version is right->left, the normal version left->right
                                overlay] (ConnectTempTail\lsConnectTempPosition.south) -- ++(0,-#2) -| (ConnectTempHead\lsConnectTempPosition.south);
    }%
}


%%%%%%%%%%%%%%%%%%%%%%%%%%%%%%%%%%%%%%%%%%%%%%%%%%%%%%%%%%%%%
%%%% Extras for use in (some) edited volumes             %%%%
%%%%%%%%%%%%%%%%%%%%%%%%%%%%%%%%%%%%%%%%%%%%%%%%%%%%%%%%%%%%%

% for chapters without abstract
\newcommand{\noabstract}{\relax}

%%%%%%%%%%%%%%%%%%%%
%%%%           %%%%%
%%%%   PLOTS   %%%%%
%%%%           %%%%%
%%%%%%%%%%%%%%%%%%%%

\newcommand{\LSfrac}[1]{{\addfontfeature{Fractions=On}#1}}% Use the fraction glyphs shipped with Libertine. The argument is n/m, where n and m are integers

% Provide a raised tie bar for diphthongs and affricates with ascenders
\newcommand{\hitie}[2]{%
\mbox{#1}%
\raisebox{.5mm}{%
͡%
}%
\mbox{#2}%
}


\newcommand{\hitier}[3][.7]{%
\mbox{#2}%
\hspace*{#1mm}%
\raisebox{.5mm}{%
͡%
}%
\hspace*{-#1mm}%
\mbox{#3}%
}


\newcommand{\hitiel}[3][.7]{%
\mbox{#2}%
\hspace*{-#1mm}%
\raisebox{.5mm}{%
͡%
}%
\hspace*{#1mm}%
\mbox{#3}%
}


% commands moved here from cgloss
\let\prmbrs=0
\def\primebars{\let\prmbrs=1}
\def\obar#1{\ifmmode#1^{0}\else#1$^{0}$\fi}  %% FIX
\def\mbar#1{\ifmmode#1^{\mathrm{max}}\else#1\textsuperscript{max}\fi}
\def\ibar#1{\ifx\prmbrs0%
                 \ifmmode\overline{\mathrm{#1}}\else$\overline{\mbox{#1}}$\fi%
            \else\ifmmode#1^{'}\else#1$^{'}$\fi\fi}
\def\iibar#1{\ifx\prmbrs0%
                  \ifmmode\overline{\overline{\mathrm{#1}}}%
                  \else$\overline{\overline{\mbox{#1}}}$\fi%
             \else #1P\fi}


\def\spec#1{[Spec,#1]} %Def. of "Specifier of #1"

% Check marks and crosses
\usepackage{pifont}
\newcommand*{\langscicheckmark}{\ding{51}}
\newcommand*{\langscicross}{\ding{55}}


\providecommand{\citegen}[2][]{\citeauthor{#2}'s (\citeyear*[#1]{#2})}
\providecommand{\citeapo}[2][]{\citeauthor{#2}' (\citeyear*[#1]{#2})}

\providecommand{\protectedex}[1]{\noindent\parbox{\linewidth}{#1}}

\providecommand{\largerpage}[1][1]{\enlargethispage{#1\baselineskip}}

% vertical space to structure tables
\providecommand{\tablevspace}{\\[-.5em]}

\providecommand{\biberror}[1]{{\color{red}#1}}

\providecommand{\lsptoprule}{\midrule\toprule}
\providecommand{\lspbottomrule}{\bottomrule\midrule}

\providecommand{\REF}[2][]{(\ref{#2#1})}


\newcommand{\glottocodes}[1]{}
\newcommand{\keywords}[1]{}

\providecommand{\ob}{{\upshape [}} %opening bracket
\providecommand{\cb}{{\upshape ]}} %closing bracket
\providecommand{\op}{{\upshape (}} %opening paren
\providecommand{\cp}{{\upshape )}} %closing paren
\providecommand{\db}{\hphantom{[}} %dummy space for [ in IMT line

% Shortcuts to langsci-affiliation styles
% These commands allow switching between two frequently used output styles
% for authors and affiliations:
% The first command does not produce any superscripted indexes for affiliations.
% Authors are output in their own line (not grouped in one line), with their
% affiliations following in the line directly below their names.
\NewDocumentCommand{\AffiliationsWithoutIndexing}{}
  {%
    \SetupAffiliations{output in groups = false,
                       separator between two = {\bigskip\\},
                       separator between multiple = {\bigskip\\},
                       separator between final two = {\bigskip\\}
                      }
  }
% This commands reverts to the standards in langsci-affiliations: authors are 
% grouped in one line, with indexes pointing to their affiliations. Affiliations
% are resolved in the line below.  
\NewDocumentCommand{\AffiliationsWithIndexing}{}
  {%
    \SetupAffiliations{output in groups = true,
                       separator between two = {~\&~},
                       separator between multiple = {,~},
                       separator between final two = {~\&~}
                      }
  }


\newcommand{\licencebox}[2][]{\parbox{#1\textwidth}{\tiny\raggedright #2}}

\section{Regression tests}
% This file contains instances of code which showed unintended behaviour in the past which is fixed now. Make sure that the unintended behaviour does not resurface

\section{Footnotes inherit font specs from calling string}

This is a text in \textit{italics\footnote{This footnote should not be in italics}} which has a footnote which should not be in italics
% \url{https://github.com/langsci/langscibook/issues/135}

\emph{Blabla\footnote{Footnote.} bla.}

\include{backmatter}
\end{document}
